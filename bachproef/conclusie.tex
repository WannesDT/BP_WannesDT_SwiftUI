%%=============================================================================
%% Conclusie
%%=============================================================================

\chapter{Conclusie}%
\label{ch:conclusie}

% TODO: Trek een duidelijke conclusie, in de vorm van een antwoord op de
% onderzoeksvra(a)g(en). Wat was jouw bijdrage aan het onderzoeksdomein en
% hoe biedt dit meerwaarde aan het vakgebied/doelgroep? 
% Reflecteer kritisch over het resultaat. In Engelse teksten wordt deze sectie
% ``Discussion'' genoemd. Had je deze uitkomst verwacht? Zijn er zaken die nog
% niet duidelijk zijn?
% Heeft het onderzoek geleid tot nieuwe vragen die uitnodigen tot verder 
%onderzoek?


\section{Analyse van Gegevensoverdrachtmethoden in SwiftUI Views}
Op basis van de voorgaande gegevens kunnen we de invloed van diverse methoden van gegevensoverdracht naar SwiftUI views op de performance, renderingfrequentie en levenscyclussen van bindings als volgt analyseren:

\subsection{Performance}
De performance wordt gemeten aan de hand van de totale CPU-gebruikstijd(hoeveelheid tijd die de CPU daadwerkelijk besteed aan het uitvoeren van een bepaalde proces) en de totale CPU-weight(hoeveelheid CPU-werk (of cycles) dat een proces heeft gebruikt):
\begin{itemize}
    \item \textit{EnvironmentObject} en \textit{ObservableObject} scoren over het algemeen goed qua CPU-gebruik, waarbij ObservableObject iets meer CPU-belasting veroorzaakt.
    \item \textit{Binding} presteert consistent slecht qua CPU-gebruik, vooral in complexe scenario's met meerdere subviews.
    \item \textit{Observable} presteert slechter dan andere methoden qua totale vernieuwingstijd in sommige tests.
\end{itemize}

\subsection{Renderingfrequentie}
De renderingfrequentie wordt gemeten aan de hand van het totaal aantal vernieuwingen en de totale duur van vernieuwingen:
\begin{itemize}
    \item \textit{EnvironmentObject} biedt een balans tussen frequentie en duur van vernieuwingen en is meestal efficiënter in de totale duur van vernieuwingen.
    \item \textit{ObservableObject} heeft de kortste totale vernieuwingstijd, wat het geschikt maakt voor scenario's waar snelle vernieuwingen cruciaal zijn.
    \item \textit{Observable} heeft een hogere totale vernieuwingstijd, wat kan leiden tot tragere updates.
\end{itemize}

\subsection{Property updates van Bindings}
De property updates van bindings beïnvloedt hoe vaak gegevens bijgewerkt worden:
\begin{itemize}
    \item \textit{Observable}: Deze methode vereist minder frequente updates, dit heeft te maken met dat observable gemaakt is voor onnodige view redraws, dit zorgt ook voor een lagere CPU belasting.
    \item \textit{Binding}: Deze methode vereist frequente updates, wat kan leiden tot hogere CPU-belasting en langere vernieuwingstijden.
    \item \textit{EnvironmentObject en ObservableObject}: Deze bieden een meer gestructureerde en efficiënte manier om gegevens over te dragen, wat leidt tot minder frequente updates.
\end{itemize}

\newpage
\section{Vergelijking van "Observable" met Andere Methoden}

De "Observable" macro is een nieuwere methode van Apple die bedoeld is om andere ``observable''-methoden voor gegevensoverdracht in SwiftUI te vervangen. Hieronder volgt een gedetailleerde vergelijking van "Observable" met \textit{Binding}, \textit{EnvironmentObject} en \textit{ObservableObject} op basis van performance, renderingfrequentie en levenscyclussen van bindings.

\subsection{Performance}
De performance wordt gemeten aan de hand van de totale CPU-gebruikstijd en het totale CPU-gewicht:
\begin{itemize}
    \item \textit{Observable} scoort beter dan \textit{Binding}, maar slechter dan \textit{EnvironmentObject} en \textit{ObservableObject} qua CPU-gebruik. 
    \item \textit{Observable} heeft een CPU-gebruikstijd van 400 ms in Test2 en 927 ms in Test3, wat beter is dan Binding maar slechter dan de andere methoden.
    \item De totale CPU-gewicht voor \textit{Observable} is 294,88 Mc in Test2 en 463,30 Mc in Test3, wat betekent dat het een zwaardere belasting heeft dan \textit{EnvironmentObject}, maar minder dan \textit{Binding} en \textit{ObservableObject}.
\end{itemize}

\subsection{Renderingfrequentie}
De renderingfrequentie wordt gemeten aan de hand van het totaal aantal vernieuwingen en de totale duur van vernieuwingen:
\begin{itemize}
    \item \textit{Observable} heeft een hogere totale vernieuwingstijd in Test2 (1,82 ms) en een gematigde score in Test3 (2,12 ms) vergeleken met andere methoden.
    \item De totale vernieuwingstijd is slechter dan \textit{EnvironmentObject} en \textit{ObservableObject}, maar beter dan \textit{Binding}.
    \item De totale vernieuwingen in Test2 en Test3 laten zien dat \textit{Observable} efficiënter is dan \textit{Binding} maar minder efficiënt dan \textit{EnvironmentObject} en \textit{ObservableObject}.
\end{itemize}


De "Observable" macro van Apple is een moderne methode voor gegevensoverdracht in SwiftUI, die is ontworpen om oudere methoden zoals \textit{Binding}, \textit{EnvironmentObject}, en \textit{ObservableObject} te vervangen of aan te vullen. Op basis van de uitgevoerde tests en analyse kunnen we de volgende conclusies trekken:

\subsection{Voordelen van ``Observable''}
\begin{itemize}
    \item \textit{Verbeterde Performance}: De ``Observable'' macro presteert beter dan \textit{Binding} in termen van CPU-gebruik en vernieuwingstijden. Dit maakt het een efficiëntere keuze voor complexere UI's waar prestaties kritisch zijn.
    \item \textit{Efficiëntere Renderingfrequentie}: De totale vernieuwingstijd voor "Observable" is lager dan die van \textit{Binding}, wat resulteert in vloeiendere en snellere UI-updates.
    \item \textit{Lagere Frequentie van Updates}: "Observable" vereist minder frequente updates dan \textit{Binding}, wat de levensduur van de binding verlengt en de CPU-belasting vermindert.
\end{itemize}

\subsection{Conclusie}
De \textit{Observable} macro biedt aanzienlijke verbeteringen ten opzichte van \textit{Binding} in specifieke scenario's waar complexere dataoverdracht vereist is in SwiftUI. Terwijl \textit{Binding} ideaal is voor primitieve types en het doorgeven van eenvoudige waarden tussen views, is \texttt{Observable} beter geschikt voor het beheren van complexere objecten en data. 

Hoewel \textit{EnvironmentObject} en \textit{ObservableObject} vaak superieure prestaties bieden in termen van CPU-gebruik en vernieuwingstijd, biedt \textit{Observable} een modernere en efficiëntere aanpak voor gegevensoverdracht die een balans vindt tussen gebruiksgemak en efficiëntie. Het gebruik van \textit{Observable} kan leiden tot betere prestaties en een responsievere gebruikersinterface in veel scenario's, waardoor het een waardevolle toevoeging is aan de toolkit van elke SwiftUI-ontwikkelaar. Het is echter belangrijk om de juiste tool voor de juiste taak te kiezen, waarbij \textit{Binding} vooral nuttig blijft voor het beheren van eenvoudige, primitieve gegevens.



\newpage
\section{Conclusie en Aanbevelingen}
\begin{itemize}
    \item \textbf{EnvironmentObject} is geschikt voor scenario's waar een balans tussen prestaties en vernieuwingstijden belangrijk is. Het biedt efficiënte gegevensoverdracht met een lage CPU-belasting.
    \item \textbf{ObservableObject} is ideaal wanneer snelle vernieuwingstijden cruciaal zijn, ondanks de hogere CPU-belasting.
    \item \textbf{Binding} moet vermeden worden in complexe UI's met meerdere subviews vanwege de hoge CPU-belasting en frequente updates.
    \item \textbf{Observable} is minder efficiënt in termen van vernieuwingstijden en moet alleen worden gebruikt wanneer specifieke voordelen van deze methode nodig zijn.
\end{itemize}

\subsection{Gebruik van Bevindingen}
Bij het ontwerpen van efficiënte gebruikersinterfaces kunnen deze bevindingen als volgt worden benut:
\begin{itemize}
    \item Kies \textbf{EnvironmentObject} voor algemene doeleinden en grotere applicaties waar meerdere subviews gegevens moeten delen.
    \item Gebruik \textbf{ObservableObject} in scenario's waar snelle responstijden cruciaal zijn, zoals bij real-time data updates.
    \item Vermijd \textbf{Binding} en \textit{Observable} in complexe of prestatiekritieke delen van de applicatie om inefficiënties te voorkomen.
\end{itemize}

Door deze methoden strategisch te kiezen op basis van de hierboven beschreven performance en efficiëntie kunnen ontwikkelaars goed presterende en responsieve gebruikersinterfaces in SwiftUI ontwerpen.



