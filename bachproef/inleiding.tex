%%=============================================================================
%% Inleiding
%%=============================================================================

\chapter{\IfLanguageName{dutch}{Inleiding}{Introduction}}%
\label{ch:inleiding}

SwiftUI is een modern framework van Apple dat een nieuwe aanpak biedt voor het ontwerpen en bouwen van gebruikersinterfaces voor iOS, macOS en andere platforms. Het maakt gebruik van een declaratieve programmeerstijl en combineert een intuïtieve syntaxis met krachtige tools, waarmee ontwikkelaars snel aantrekkelijke en responsieve apps kunnen maken. UIKit, het oudere framework, was niet declaratief en daardoor niet zo modern als SwiftUI. Met SwiftUI kunnen ontwikkelaars sneller en responsievere apps maken, wat bijdraagt aan een verbeterde gebruikerservaring. Met de groeiende populariteit van SwiftUI en het belang van efficiënte en reactieve gegevens overdracht met een minumum aan side effects in apps, is er behoefte aan diepgaand onderzoek naar de impact van rendercycli op prestaties en rendering.

Dit onderzoek richt zich op het afbakenen van dit onderwerp door te onderzoeken hoe verschillende overdrachtsmethoden in SwiftUI de prestaties van applicaties beïnvloeden. De efficiënte verwerking van gegevensoverdracht is cruciaal voor het leveren van een soepele gebruikerservaring, en daarom is het belangrijk om te begrijpen welke methoden het meest effectief zijn.

De probleemstelling van dit onderzoek is het identificeren van de meest efficiënte methode voor gegevensoverdracht in SwiftUI-toepassingen. Het onderzoeksdoel is om de impact van verschillende rendercycli op het geheugengebruik, het CPU-gebruik en de laadtijd van SwiftUI-weergaven te evalueren.

\section{\IfLanguageName{dutch}{Probleemstelling}{Problem Statement}}%
\label{sec:probleemstelling}

%
%Uit je probleemstelling moet duidelijk zijn dat je onderzoek een meerwaarde heeft voor een concrete doelgroep. De doelgroep moet goed gedefinieerd en afgelijnd zijn. Doelgroepen als ``bedrijven,'' ``KMO's'', systeembeheerders, enz.~zijn nog te vaag. Als je een lijstje kan maken van de personen/organisaties die een meerwaarde zullen vinden in deze bachelorproef (dit is eigenlijk je steekproefkader), dan is dat een indicatie dat de doelgroep goed gedefinieerd is. Dit kan een enkel bedrijf zijn of zelfs één persoon (je co-promotor/opdrachtgever).
De diversiteit aan methoden die beschikbaar zijn voor het doorgeven van gegevens in SwiftUI roept vragen op over de optimale keuze voor specifieke gebruiksscenario's. Het gebrek aan diepgaande vergelijkende studies maakt het moeilijk om beslissingen te nemen bij het ontwerpen van efficiënte gebruikersinterfaces en architecturen. Apple geeft hierover namelijk weinig informatie en schermt achterliggende code af.


\section{\IfLanguageName{dutch}{Onderzoeksvraag}{Research question}}%
\label{sec:onderzoeksvraag}
In hoeverre beïnvloeden diverse methoden van gegevensoverdracht naar SwiftUI views de performance, renderingfrequentie en levenscyclussen van views? Hoe kunnen deze bevindingen worden benut om weloverwogen keuzes te maken bij het ontwerpen van efficiënte gebruikersinterfaces?

\section{\IfLanguageName{dutch}{Onderzoeksdoelstelling}{Research objective}}%
\label{sec:onderzoeksdoelstelling}

%Wat is het beoogde resultaat van je bachelorproef? Wat zijn de criteria voor succes? Beschrijf die zo concreet mogelijk. Gaat het bv.\ om een proof-of-concept, een prototype, een verslag met aanbevelingen, een vergelijkende studie, enz.

Deze bachelorproef beoogt een diepgaand inzicht te geven van rendercycli in SwiftUI en streeft ernaar om ontwikkelaars te ondersteunen in weloverwogen keuzes te maken voor het optimaliseren van de prestaties en rendering van hun iOS-applicaties te optimaliseren. Door een vergelijkende analyse van verschillende methoden voor gegevensoverdracht worden duidelijke aanbevelingen gedaan voor effectieve UI-implementaties in SwiftUI.
\section{\IfLanguageName{dutch}{Opzet van deze bachelorproef}{Structure of this bachelor thesis}}%
\label{sec:opzet-bachelorproef}

% Het is gebruikelijk aan het einde van de inleiding een overzicht te
% geven van de opbouw van de rest van de tekst. Deze sectie bevat al een aanzet
% die je kan aanvullen/aanpassen in functie van je eigen tekst.

De rest van deze bachelorproef is als volgt opgebouwd:

In Hoofdstuk~\ref{ch:stand-van-zaken} wordt een overzicht gegeven van de stand van zaken binnen het onderzoeksdomein, op basis van een literatuurstudie.

In Hoofdstuk~\ref{ch:methodologie} wordt de methodologie toegelicht en worden de gebruikte onderzoekstechnieken besproken om een antwoord te kunnen formuleren op de onderzoeksvragen.

% TODO: Vul hier aan voor je eigen hoofstukken, één of twee zinnen per hoofdstuk
In Hoofdstuk~\ref{ch:ruweresultaten} worden de ruwe resultaten opgelijst en kort besproken hoe ze bekomen werden.

In Hoofdstuk~\ref{ch:resultaten} worden de resultaten toegelicht en besproken om een antwoord te kunnen formuleren op de onderzoeksvragen.

In Hoofdstuk~\ref{ch:conclusie}, tenslotte, wordt de conclusie gegeven en een antwoord geformuleerd op de onderzoeksvragen. Daarbij wordt ook een aanzet gegeven voor toekomstig onderzoek binnen dit domein.