%%=============================================================================
%% Samenvatting
%%=============================================================================

% TODO: De "abstract" of samenvatting is een kernachtige (~ 1 blz. voor een
% thesis) synthese van het document.
%
% Een goede abstract biedt een kernachtig antwoord op volgende vragen:
%
% 1. Waarover gaat de bachelorproef?
% 2. Waarom heb je er over geschreven?
% 3. Hoe heb je het onderzoek uitgevoerd?
% 4. Wat waren de resultaten? Wat blijkt uit je onderzoek?
% 5. Wat betekenen je resultaten? Wat is de relevantie voor het werkveld?
%
% Daarom bestaat een abstract uit volgende componenten:
%
% - inleiding + kaderen thema
% - probleemstelling
% - (centrale) onderzoeksvraag
% - onderzoeksdoelstelling
% - methodologie
% - resultaten (beperk tot de belangrijkste, relevant voor de onderzoeksvraag)
% - conclusies, aanbevelingen, beperkingen
%
% LET OP! Een samenvatting is GEEN voorwoord!

%%---------- Nederlandse samenvatting -----------------------------------------
%
% TODO: Als je je bachelorproef in het Engels schrijft, moet je eerst een
% Nederlandse samenvatting invoegen. Haal daarvoor onderstaande code uit
% commentaar.
% Wie zijn bachelorproef in het Nederlands schrijft, kan dit negeren, de inhoud
% wordt niet in het document ingevoegd.

\IfLanguageName{english}{%
\selectlanguage{dutch}
\chapter*{Samenvatting}
\lipsum[1-4]
\selectlanguage{english}
}{}

%%---------- Samenvatting -----------------------------------------------------
% De samenvatting in de hoofdtaal van het document

\chapter*{\IfLanguageName{dutch}{Samenvatting}{Abstract}}
Deze bachelorproef richt zich op een vergelijkende studie van rendercyclussen in SwiftUI en onderzoekt de impact op performance, rendering en lifecycles. SwiftUI is een populair framework voor het ontwikkelen van apps, en het is belangrijk om inzicht te krijgen in de efficiëntie en prestaties van rendercyclussen voor de ontwikkeling van responsieve applicaties.

De centrale onderzoeksvraag van deze studie is: In hoeverre beïnvloeden diverse methoden van gegevensoverdracht naar SwiftUI views de performance, renderingfrequentie en levenscyclussen van bindings, en hoe kunnen deze bevindingen worden benut om weloverwogen keuzes te maken bij het ontwerpen van efficiënte gebruikersinterfaces?

Om deze vraag te beantwoorden, is een serie testen uitgevoerd waarbij verschillende methoden van dataoverdracht in SwiftUI zijn getest. Elke methode is geëvalueerd op het aantal keren dat een view werd ververst tijdens dataoverdracht, evenals het gemiddelde geheugengebruik, CPU-gebruik en laadtijd van de views.

De resultaten van het onderzoek tonen duidelijke verschillen in performantie tussen de verschillende methoden van dataoverdracht. Sommige methoden bleken consistenter en efficiënter te zijn in termen van snelheid en resourcegebruik, terwijl andere methoden meer variatie toonden in hun prestaties. Deze bevindingen bieden waardevolle inzichten voor het optimaliseren van SwiftUI-applicaties.

De conclusies van deze studie dragen bij aan het werkveld door ontwikkelaars te helpen bij het kiezen van de meest geschikte methoden voor dataoverdracht in SwiftUI-applicaties. Dit onderzoek kan bijdragen aan de verbetering van de algehele gebruikerservaring en de ontwikkeling van efficiëntere en responsievere apps.
