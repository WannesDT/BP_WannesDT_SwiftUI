%%=============================================================================
%% Samenvatting
%%=============================================================================

% TODO: De "abstract" of samenvatting is een kernachtige (~ 1 blz. voor een
% thesis) synthese van het document.
%
% Een goede abstract biedt een kernachtig antwoord op volgende vragen:
%
% 1. Waarover gaat de bachelorproef?
% 2. Waarom heb je er over geschreven?
% 3. Hoe heb je het onderzoek uitgevoerd?
% 4. Wat waren de resultaten? Wat blijkt uit je onderzoek?
% 5. Wat betekenen je resultaten? Wat is de relevantie voor het werkveld?
%
% Daarom bestaat een abstract uit volgende componenten:
%
% - inleiding + kaderen thema
% - probleemstelling
% - (centrale) onderzoeksvraag
% - onderzoeksdoelstelling
% - methodologie
% - resultaten (beperk tot de belangrijkste, relevant voor de onderzoeksvraag)
% - conclusies, aanbevelingen, beperkingen
%
% LET OP! Een samenvatting is GEEN voorwoord!

%%---------- Nederlandse samenvatting -----------------------------------------
%
% TODO: Als je je bachelorproef in het Engels schrijft, moet je eerst een
% Nederlandse samenvatting invoegen. Haal daarvoor onderstaande code uit
% commentaar.
% Wie zijn bachelorproef in het Nederlands schrijft, kan dit negeren, de inhoud
% wordt niet in het document ingevoegd.

\IfLanguageName{english}{%
\selectlanguage{dutch}
\chapter*{Samenvatting}
\lipsum[1-4]
\selectlanguage{english}
}{}

%%---------- Samenvatting -----------------------------------------------------
% De samenvatting in de hoofdtaal van het document

\chapter*{\IfLanguageName{dutch}{Samenvatting}{Abstract}}

Deze bachelorproef richt zich op een vergelijkende studie van rendercycli in SwiftUI en onderzoekt de impact op prestaties, rendering en levenscycli. SwiftUI is een populair framework voor het ontwikkelen van apps, en het is belangrijk om de efficiëntie en prestaties van rendercycli voor het ontwikkelen van responsieve applicaties te begrijpen.

De centrale onderzoeksvraag van dit onderzoek is: In hoeverre beïnvloeden verschillende methoden van gegevensoverdracht naar SwiftUI-views de prestaties, renderfrequentie en levenscycli van bindings, en hoe kunnen deze bevindingen worden gebruikt om weloverwogen keuzes te maken bij het ontwerpen van efficiënte gebruikersinterfaces?

Om deze vraag te beantwoorden, is een reeks tests uitgevoerd waarbij verschillende methoden voor gegevensoverdracht in SwiftUI zijn getest. Elke methode werd geëvalueerd op het aantal keren dat een weergave werd ververst tijdens de gegevensoverdracht, evenals het  CPU-gebruik en laadtijd van de weergaven.

De resultaten van het onderzoek laten duidelijke prestatieverschillen zien tussen verschillende methoden van gegevensoverdracht. Sommige methoden bleken consistenter en efficiënter in termen van snelheid, terwijl andere methoden meer variatie in hun prestaties lieten zien. Deze bevindingen bieden waardevolle inzichten voor het optimaliseren van SwiftUI toepassingen.

De conclusies van dit onderzoek dragen bij aan het werkveld door ontwikkelaars te helpen bij het kiezen van de meest geschikte methoden voor gegevensoverdracht in SwiftUI-toepassingen. Dit onderzoek kan helpen om de algehele gebruikerservaring te verbeteren en efficiëntere en responsievere apps te ontwikkelen.