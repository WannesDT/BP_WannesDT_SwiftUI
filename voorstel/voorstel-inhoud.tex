%---------- Inleiding ---------------------------------------------------------

\section{Introductie}%
\label{sec:introductie}

In deze bachelorproef word een grondige verkenning van de rendercyclussen in SwiftUI gedaan, met een specifieke focus op de impact ervan op prestaties, rendering en de levenscyclus van bindings. SwiftUI, een innovatief framework voor het bouwen van gebruikersinterfaces in Swift, heeft de ontwikkeling van iOS-applicaties getransformeerd door de introductie van declaratieve syntax en reactieve programmeerprincipes. In deze bachelorproef willen we de verschillende methoden voor gegevensoverdracht naar SwiftUI-views identificeren en analyseren, met als ultiem doel ontwikkelaars inzicht te verschaffen in de meest efficiënte benaderingen voor het bereiken van de beste UI-prestaties.

\subsection{Doelgroep}
Deze studie richt zich op softwareontwikkelaars, met een specifieke interesse in iOS-applicatie ontwikkeling en SwiftUI. Het biedt een diepgaand inzicht in de keuzemogelijkheden voor data-overdracht en de gevolgen ervan op de algehele gebruikersinterface.

\subsection{Probleemstelling}
De diversiteit aan beschikbare methoden voor het doorgeven van gegevens in SwiftUI roept vragen op over de optimale keuze voor specifieke gebruiksscenario's. Het gebrek aan diepgaande vergelijkende studies maakt het voor ontwikkelaars moeilijk om beslissingen te nemen bij het ontwerpen van efficiënte gebruikersinterfaces.

\subsection{Centrale onderzoeksvraag}
In hoeverre beïnvloeden diverse methoden van gegevensoverdracht naar SwiftUI views de performance, renderingfrequentie en levenscyclussen van bindings, en hoe kunnen deze bevindingen worden benut om weloverwogen keuzes te maken bij het ontwerpen van efficiënte gebruikersinterfaces?

\subsection{Onderzoeksdoelstelling}
Deze bachelorproef beoogt een diepgaand begrip van de rendercyclussen in SwiftUI en streeft ernaar om ontwikkelaars te voorzien van concrete richtlijnen voor het optimaliseren van de prestaties en rendering van hun iOS-applicaties. Door een vergelijkende analyse van de verschillende dataoverdrachtsmethoden wordt beoogd heldere aanbevelingen te formuleren voor effectieve UI-implementaties in SwiftUI.


%---------- Stand van zaken ---------------------------------------------------

\section{Literatuurstudie}%
\label{sec:state-of-the-art}

Uit al de volgende bronnen kunnen we informatie halen omtrent verschillende manieren van data doorgeven in SwiftUI. Zo zijn er deze 6 Veelgebruikte methoden:
\newline\textbf{@State:}
\begin{itemize}
    \item {Lokale en eenvoudige gegevensopslag die het gedrag van een view beïnvloeden.}
    \item {Wordt meestal gebruikt voor primitieve gegevenstypes zoals integers, strings en boole-\newline ans.}
\end{itemize}
\textbf{@Binding:}
\begin{itemize}
    \item {Het doorgeven van gegevens tussen de ouder-view en de onderliggende view.}
    \item {De waarde wordt doorgegeven als een binding, waardoor de onderliggende view de waarde kan wijzigen.}
\end{itemize}
\textbf{@ObservedObject:}
\begin{itemize}
    \item {Het doorgeven van gegevens die door een externe klasse worden beheerd en waarmee de view wordt geïnformeerd over wijzigingen.}
    \item {Vereist het gebruik van een ObservableObject klasse.}
\end{itemize}
\textbf{@StateObject:}
\begin{itemize}
    \item {Vergelijkbaar met @ObservedObject, maar met een levenscyclus die is gebonden aan de view.}
    \item {Wordt geïnitialiseerd en beheerd door de view en kan worden gebruikt voor het bijhouden van de status van de view.}
\end{itemize}
\textbf{@Observable:}
\begin{itemize}
    \item {Een vervanging van vele oude methodes zoals State en Environment}
    \item {Apple claimed dat deze nieuwe macro performanter is}
\end{itemize}




De volgende macro's zijn niet vergelijkbaar met de andere omdat deze een specifiek doel hebben maar zijn nog steeds de moeite om te vermelden:
\newline\textbf{@EnvironmentObject:}
\begin{itemize}
    \item {Het doorgeven van gegevens die door een externe klasse worden beheerd, toegankelijk voor de hele app zonder expliciete doorgifte.}
    \item {Vereist het gebruik van een ObservableObject-klasse en wordt vaak gebruikt voor gedeelde gegevens zoals gebruikersauthenticatie.}
\end{itemize}
\textbf{@Binding in init:}
\begin{itemize}
    \item {Het initialiseren van een binding tijdens de init-fase van een view.}
    \item {Maakt het mogelijk om een binding te maken en door te geven bij de creatie van de view.}
\end{itemize}

% Dit is vloeiende tekst
% Uit enkele relevante studies kunnen we concluderen dat er veel vraag en verwarring is rond de correcte manier van data doorgeven aan SwiftUI views. Verschillende bronnen benadrukken dit thema en bieden waardevolle inzichten. Het artikel "Passing Data to SwiftUI Views"~\autocite{SwiftDevJournal} identificeert expliciet drie methoden voor gegevensoverdracht aan views, wat cruciale informatie verschaft voor een dieper begrip van dit onderwerp. Bovendien onthult de studie "Deep Inside Views, State and Performance in SwiftUI"~\autocite{Long2020} dat veel bedrijven en ontwikkelaars nog steeds vasthouden aan de denkwijze van UIKit, wat potentieel verwarrend kan zijn bij het doorgeven van gegevens naar views in SwiftUI. Deze bevindingen benadrukken het belang van een helder begrip van de verschillen tussen UIKit en SwiftUI views. Binnen het kader van prestatieoptimalisatie, zoals besproken in "SwiftUI Performance Tuning: Tips and Tricks"~\autocite{Amisha2022}, wordt aanbevolen om de annotaties @Binding en @State te gebruiken voor efficiënte gegevensoverdracht, aangezien deze methoden een betere prestatie laten zien in vergelijking met @ObservedObject en @EnvironmentObject.

\textbf{Passing Data to SwiftUI Views~\autocite{SwiftDevJournal}}

In dit artikel worden waardevolle inzichten gepresenteerd met betrekking tot diverse benaderingen voor het doorgeven van gegevens naar SwiftUI-views. Het artikel identificeert expliciet drie methoden voor gegevensoverdracht aan views en biedt daarmee cruciale informatie voor ons begrip van dit onderwerp.

\textbf{Deep Inside Views, State and Performance in SwiftUI~\autocite{Long2020}}

Deze bron onthult dat veel bedrijven en softwareontwikkelaars nog steeds vasthouden aan de denkwijze van UIKit in plaats van de nieuwe benadering van SwiftUI-views. Het is van belang te benadrukken dat UIKit en SwiftUI views twee verschillende concepten zijn. Deze neiging om vast te houden aan de oudere UIKit-mentaliteit kan potentiële verwarring veroorzaken bij het doorgeven van gegevens naar views in SwiftUI.

\textbf{SwiftUI Performance Tuning: Tips and Tricks\newline ~\autocite{Amisha2022}}

Uit deze bron blijkt dat het gebruik van de annotaties @Binding en @State voor het doorgeven van gegevens efficiënter is dan het gebruik van @ObservedObject en @EnvironmentObject.


\textbf{Pro iPhone Development with SwiftUI~\autocite{WangWallace2022PIDw}}

Uit dit onderzoek kunnen we halen dat State variabelen handig zijn voor het opslaan van afzonderlijke data blocks zoals integers (Int), decimale getallen (Double of Float), of tekst (String). Bindingsvariabelen zijn nuttig voor het aanpassen van Staatsvariabelen vanuit een andere structuur dan de structuur die de State variabele heeft gedefinieerd.

\textbf{Asynchronous Programming with SwiftUI and Combine: Functional Programming to Build UIs on Apple Platforms~\autocite{friese2023state}}

Uit dit boek kunnen we ook halen dat een @StateObject, @ObservedObject, of @EnvironmentObject, dus een ObservableObject gererenderd worden wanneer SwiftUi een verandering registreert binnen het State model. Dit houd in dat SwiftUI de actuele data in sync houd met de data die weergegeven word in de View. Ook kunnen we nuttige info vergaren uit dit boek over @StateObject. Ze beweren dat SwiftUi de lifecycle van het onderliggend object beheerd en zo er ook voor zou zorgen dat het maar 1 keer aangemaakt zou worden, Zelfs als SwiftUI de hele View opnieuw zou moeten maken. In tegenstelling tot @StateObject wordt @ObservedObject elke keer herbouwt nadat een SwiftUi View wordt herbouwd.

Uit dit artikel kunnen we ook leren dat @EnvironmentObject geen veilige manier is van dataoverdracht omdat deze niet controleert of er wel een ObservableObject word geïnjecteerd. Hierdoor kan de applicatie onverwachts crashen.



% Voor literatuurverwijzingen zijn er twee belangrijke commando's:
% \autocite{KEY} => (Auteur, jaartal) Gebruik dit als de naam van de auteur
%   geen onderdeel is van de zin.
% \textcite{KEY} => Auteur (jaartal)  Gebruik dit als de auteursnaam wel een
%   functie heeft in de zin (bv. ``Uit onderzoek door Doll & Hill (1954) bleek
%   ...'')



%---------- Methodologie ------------------------------------------------------
\section{Methodologie}%
\label{sec:methodologie}

\subsection{Onderzoekstechniek}
Dit onderzoek zal hoofdzakelijk gebruikmaken van een vergelijkende studie om de impact van verschillende methoden voor gegevensoverdracht naar SwiftUI views te analyseren. De primaire focus ligt op het vergelijken van de prestaties, renderingfrequentie en levenscycli van bindings bij het toepassen van verschillende benaderingen. De vergelijkende studie omvat het oplijsten van alle verschillende manieren om data door te geven aan SwiftUI views en langs elkaar zetten wat ieder zijn impact op performance is, hoeveel keer views gererendered  worden en wat de lifecycles van deze bindings zijn. 

\subsection{Tools}
\textbf{Xcode en SwiftUI Framework}

De ontwikkelingsomgeving Xcode, met het SwiftUI framework, zal worden gebruikt om praktische implementaties uit te voeren en de impact van verschillende dataoverdrachtsmethoden te meten.

\textbf{XCode Profiler}

Met dit instrument kunnen we performantie gaan tracken door middel van CPU usage en memory problems.

\textbf{XCTest framework}

Dit framework zorgt ervoor dat we performantie kunnen waarnemen in swiftUI.

\textbf{Google en Apple Developer Website}

Deze applicaties gaan we gebruiken om info te vergaren over SwiftUI en de werking van Swift.


\subsection{Tijdschatting}
De totale geschate tijd die nodig is zijn 15 weken waarbij ruimte is voor mogelijke aanpassingen afhankelijk van de specifieke behoeften van elke fase.

\textbf{Voorstel}
\begin{itemize}
    \item {2 Weken}
    \item {Voorstel uitschrijven in een LaTeX document}
    \item {Opzetten van github repo}
\end{itemize}

\textbf{Implementatie en metingen}
\begin{itemize}
    \item {8 Weken}
    \item {Gedocumenteerde implementatie van de onderzoeksopdracht}
    \item {Meetresultaten waarnemen van presentaties en rendering}
\end{itemize}

\textbf{Vergelijkende analyse}
\begin{itemize}
    \item {1 Week}
    \item {Analyse van de resultaten en vergelijken}
\end{itemize}

\textbf{Schrijven thesis}
\begin{itemize}
    \item {4 Weken}
    \item {Voltooide thesis met inbegrip van introductie, literatuurstudie, methodologie, resultaten, discussie, conclusies}
\end{itemize}



%---------- Verwachte resultaten ----------------------------------------------
\section{Verwacht resultaat, conclusie}
\label{sec:verwachte_resultaten}


\subsection{Verwacht resultaat}
We verwachten uit de vergelijkende studie een duidelijk inzicht te verkrijgen in de impact van verschillende methoden voor gegevensoverdracht naar SwiftUI views op prestaties, renderingfrequentie en levenscycli van bindings. De metingen zullen waarschijnlijk resulteren in kwantificeerbare gegevens die de efficiëntie van de @Binding, @State, @ObservedObject en @EnvironmentObject in verschillende scenario's illustreren.

\subsection{Conclusie}
We verwachten dat de @Binding en @State het best zullen uitkomen wat betreft prestaties en renderingfrequentie, terwijl @ObservedObject en @EnvironmentObject mogelijk meer resources vereisen. De levenscyclusgrafiek van bindings zal waarschijnlijk aangeven dat @Binding en @State bindings een meer voorspelbare levensduur hebben in vergelijking met @ObservedObject en @EnvironmentObject.

\subsection{Meerwaarde doelgroep}
De doelgroep, bestaande uit SwiftUI ontwikkelaars en IOS applicatie ontwikkelaars, zal profiteren van concrete inzichten in welke gegevensoverdrachtsmethoden de voorkeur verdienen in verschillende scenario's. Door de resultaten van dit onderzoek kunnen zij goede keuzes maken bij het ontwerpen van efficiënte gebruikersinterfaces in SwiftUI. De meerwaarde van deze bachelorproef ligt in het bieden van praktische richtlijnen die direct toepasbaar zijn in de dagelijkse praktijk van iOS applicatieontwikkeling, en zo bijdragen aan de optimalisatie van SwiftUI gebaseerde projecten.


