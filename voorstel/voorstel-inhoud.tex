%---------- Inleiding ---------------------------------------------------------

\section{Introductie}%
\label{sec:introductie}

In deze bachelorproef word een grondige verkenning van de rendercyclussen in SwiftUI gedaan, met een specifieke focus op de impact ervan op prestaties, rendering en de levenscyclus van bindings. SwiftUI, een innovatief framework voor het bouwen van gebruikersinterfaces in Swift, heeft de ontwikkeling van iOS-applicaties getransformeerd door de introductie van declaratieve syntax en reactieve programmeerprincipes. In deze bachelorproef willen we de verschillende methoden voor gegevensoverdracht naar SwiftUI-views identificeren en analyseren, met als ultiem doel ontwikkelaars inzicht te verschaffen in de meest efficiënte benaderingen voor het bereiken van de beste UI-prestaties.

\subsection{Doelgroep}
Deze studie richt zich op softwareontwikkelaars, met een specifieke interesse in iOS-applicatie ontwikkeling en SwiftUI. Het biedt een diepgaand inzicht in de keuzemogelijkheden voor data-overdracht en de gevolgen ervan op de algehele gebruikersinterface.

\subsection{Probleemstelling}
De diversiteit aan beschikbare methoden voor het doorgeven van gegevens in SwiftUI roept vragen op over de optimale keuze voor specifieke gebruiksscenario's. Het gebrek aan diepgaande vergelijkende studies maakt het voor ontwikkelaars moeilijk om beslissingen te nemen bij het ontwerpen van efficiënte gebruikersinterfaces.

\subsection{Centrale onderzoeksvraag}
In hoeverre beïnvloeden diverse methoden van gegevensoverdracht naar SwiftUI views de performance, renderingfrequentie en levenscyclussen van bindings, en hoe kunnen deze bevindingen worden benut om weloverwogen keuzes te maken bij het ontwerpen van efficiënte gebruikersinterfaces?

\subsection{Onderzoeksdoelstelling}
Deze bachelorproef beoogt een diepgaand begrip van de rendercyclussen in SwiftUI en streeft ernaar om ontwikkelaars te voorzien van concrete richtlijnen voor het optimaliseren van de prestaties en rendering van hun iOS-applicaties. Door een vergelijkende analyse van de verschillende dataoverdrachtsmethoden wordt beoogd heldere aanbevelingen te formuleren voor effectieve UI-implementaties in SwiftUI.


%---------- Stand van zaken ---------------------------------------------------

\section{Literatuurstudie}%
\label{sec:state-of-the-art}

\subsection{Passing Data to SwiftUI Views~\autocite{SwiftDevJournal}}
In dit artikel worden waardevolle inzichten gepresenteerd met betrekking tot diverse benaderingen voor het doorgeven van gegevens naar SwiftUI-views. Het artikel identificeert expliciet drie methoden voor gegevensoverdracht aan views en biedt daarmee cruciale informatie voor ons begrip van dit onderwerp.

\subsection{Deep Inside Views, State and Performance in SwiftUI~\autocite{Long2020}}
Deze bron onthult dat veel bedrijven en softwareontwikkelaars nog steeds vasthouden aan de denkwijze van UIKit in plaats van de nieuwe benadering van SwiftUI-views. Het is van belang te benadrukken dat UIKit en SwiftUI views twee verschillende concepten zijn. Deze neiging om vast te houden aan de oudere UIKit-mentaliteit kan potentiële verwarring veroorzaken bij het doorgeven van gegevens naar views in SwiftUI.

\subsection{SwiftUI Performance Tuning: Tips and Tricks~\autocite{Amisha2022}}
Uit deze bron blijkt dat het gebruik van de annotaties @Binding en @State voor het doorgeven van gegevens efficiënter is dan het gebruik van @ObservedObject en @EnvironmentObject.



% Voor literatuurverwijzingen zijn er twee belangrijke commando's:
% \autocite{KEY} => (Auteur, jaartal) Gebruik dit als de naam van de auteur
%   geen onderdeel is van de zin.
% \textcite{KEY} => Auteur (jaartal)  Gebruik dit als de auteursnaam wel een
%   functie heeft in de zin (bv. ``Uit onderzoek door Doll & Hill (1954) bleek
%   ...'')



%---------- Methodologie ------------------------------------------------------
\section{Methodologie}%
\label{sec:methodologie}

Hier beschrijf je hoe je van plan bent het onderzoek te voeren. Welke onderzoekstechniek ga je toepassen om elk van je onderzoeksvragen te beantwoorden? Gebruik je hiervoor literatuurstudie, interviews met belanghebbenden (bv.~voor requirements-analyse), experimenten, simulaties, vergelijkende studie, risico-analyse, PoC, \ldots?

Valt je onderwerp onder één van de typische soorten bachelorproeven die besproken zijn in de lessen Research Methods (bv.\ vergelijkende studie of risico-analyse)? Zorg er dan ook voor dat we duidelijk de verschillende stappen terug vinden die we verwachten in dit soort onderzoek!

Vermijd onderzoekstechnieken die geen objectieve, meetbare resultaten kunnen opleveren. Enquêtes, bijvoorbeeld, zijn voor een bachelorproef informatica meestal \textbf{niet geschikt}. De antwoorden zijn eerder meningen dan feiten en in de praktijk blijkt het ook bijzonder moeilijk om voldoende respondenten te vinden. Studenten die een enquête willen voeren, hebben meestal ook geen goede definitie van de populatie, waardoor ook niet kan aangetoond worden dat eventuele resultaten representatief zijn.

Uit dit onderdeel moet duidelijk naar voor komen dat je bachelorproef ook technisch voldoen\-de diepgang zal bevatten. Het zou niet kloppen als een bachelorproef informatica ook door bv.\ een student marketing zou kunnen uitgevoerd worden.

Je beschrijft ook al welke tools (hardware, software, diensten, \ldots) je denkt hiervoor te gebruiken of te ontwikkelen.

Probeer ook een tijdschatting te maken. Hoe lang zal je met elke fase van je onderzoek bezig zijn en wat zijn de concrete \emph{deliverables} in elke fase?

%---------- Verwachte resultaten ----------------------------------------------
\section{Verwacht resultaat, conclusie}%
\label{sec:verwachte_resultaten}

Hier beschrijf je welke resultaten je verwacht. Als je metingen en simulaties uitvoert, kan je hier al mock-ups maken van de grafieken samen met de verwachte conclusies. Benoem zeker al je assen en de onderdelen van de grafiek die je gaat gebruiken. Dit zorgt ervoor dat je concreet weet welk soort data je moet verzamelen en hoe je die moet meten.

Wat heeft de doelgroep van je onderzoek aan het resultaat? Op welke manier zorgt jouw bachelorproef voor een meerwaarde?

Hier beschrijf je wat je verwacht uit je onderzoek, met de motivatie waarom. Het is \textbf{niet} erg indien uit je onderzoek andere resultaten en conclusies vloeien dan dat je hier beschrijft: het is dan juist interessant om te onderzoeken waarom jouw hypothesen niet overeenkomen met de resultaten.

